\documentclass[11pt]{article}

\usepackage{ifthen}
\usepackage{cprotect}

\newboolean{Solutions}
\setboolean{Solutions}{true}
\newcommand{\IfSolnElse}[2]{\ifthenelse{\boolean{Solutions}}{#1}{#2}}

\topmargin = 0in
\oddsidemargin = 0in
\evensidemargin = \oddsidemargin
\textwidth = 6.5in
\textheight = 9in

\title{Midterm Exam \#1}

\author{CS97: Principles and Practices of Computing}

\date{Wednesday, October 25, 2017}

\begin{document}  
\maketitle
\thispagestyle{empty}
\pagestyle{empty}

  \begin{flushright}
\fbox{  
\begin{tabular}{ll}
  1. \hspace{.5in} & 2. \hspace{.5in} \\\\\\
  3.& 4. \\\\\\
\end{tabular}
}
\end{flushright}

\vspace{.5in}

Name:$\underline{\mbox{\hspace{4in}}}$

\vspace{.5in}

ID:$\underline{\mbox{\hspace{4.2in}}}$

\vspace{0.5in}

Rules of the game:

\begin{itemize}
  \item {\bf Write your name and ID number above.}
\item The exam is closed-book and closed-notes.

\item Please write your answers
directly on the exam.  Do not turn in anything else.

% \item Obey our usual OCaml style rules.

% \item Except where explicitly disallowed, you can write any number of
 % helper functions that you need.

% \item If you have any questions, please ask.  
  
\item The exam ends promptly at 3:50pm.
  
\item Read questions carefully.  Understand a question before you
  start writing. %   {\em Note:  Some multiple-choice
    % questions ask for a single answer, while others ask for all
    % appropriate answers.} 

\item Relax!
\end{itemize}

\begin{enumerate}

\pagebreak

\item (2 points each)
Consider this function, where {\tt x}, {\tt y}, and {\tt z} are integers:

THIS EXAMPLE IS CONFUSING BECAUSE WE HAVE NEVER TALKED ABOUT RE-ASSIGNING VARIABLES!  AVOID NEXT TIME.

\begin{verbatim}
def secret(x,y,z):
    if x < y:
        answer = 0
    if y < z:
        answer = 1
    elif z < x:
        answer = 2
    else:
        answer = 3
    return answer
\end{verbatim}

\bigskip

\begin{enumerate}
\item What is the result of {\tt secret(1,2,3)}?
\begin{enumerate}
  \item 0
  \item 1
  \item 2
  \item 3
\end{enumerate}

\IfSolnElse{ii}{\vspace{.5in}}

\item What is the result of {\tt secret(2,2,1)}?
\begin{enumerate}
  \item 0
  \item 1
  \item 2
  \item 3
\end{enumerate}

\IfSolnElse{iii}{\vspace{.5in}}

\item What is the result of {\tt secret(3,1,2)}?
\begin{enumerate}
  \item 0
  \item 1
  \item 2
  \item 3
\end{enumerate}

\IfSolnElse{ii}{\vspace{.5in}}

\end{enumerate}

\pagebreak

\item (5 points) The federal income tax plan currently being considered in Congress imposes a 12\% tax on the first \$37,500 of a person's income, a 25\% tax on all income above \$37,500 and less than or equal to \$112,500, and a 35\% tax on all income above \$112,500.  For example, someone earning \$30,000 would owe 30,000 * 0.12 = \$3600 in taxes, while someone earning \$40,000 would owe (37,500 * 0.12) + (2500 * 0.25) = \$5125 in taxes.  Implement the function {\tt taxes} that is declared below, which takes an income (a number) as an argument and returns the taxes owed.

\bigskip

\begin{verbatim}
def taxes(income):
    # your code goes here
\end{verbatim}

\cprotect
\IfSolnElse{
\begin{verbatim}
def taxes(income):
    if income <= 37500:
        return income * 0.12
    elif income <= 112500:
        return 37500 * 0.12 + (income - 37500) * 0.25
    else:
        return 37500 * 0.12 + (112500 - 37500) * 0.25 + (income - 112500) * 0.35
\end{verbatim}
}{}

\pagebreak

\item (2 points each) Consider this function, where {\tt lst} is a list and {\tt n} is an integer:

\begin{verbatim}
def mystery(lst, n):
    if n == 0:
        return lst
    elif lst == []:
        return lst
    else:
        return mystery(lst[1:], n-1) + [lst[0]]
\end{verbatim}

\bigskip

\begin{enumerate}
\item What does {\tt mystery([1,3,5,7,9], 2)} return?

\IfSolnElse{{\tt [5,7,9,3,1]}}{\vspace{1.5in}}

\item How many times is {\tt mystery} called during the execution of {\tt mystery([1,3,5,7,9], 2)}, including the initial call to {\tt mystery}?

\IfSolnElse{3}{\vspace{1.5in}}

\item What are the two argument lists to the first concatenation operation (i.e., execution of the {\tt +} operator) that happens during the execution of {\tt mystery([1,3,5,7,9], 2)}?

\IfSolnElse{{\tt [5,7,9]} and {\tt [3]}}{}

\end{enumerate}

\pagebreak

\item (5 points) In mathematics, the \emph{dot product} of two vectors of numbers \verb+[a1,a2,...,aN]+ and \verb+[b1,b2,...,bN]+ is defined as \verb%a1*b1 + a2*b2 + ... + aN*bN%.  Implement the function {\tt dotProduct} declared below, where Python lists are used to represent the vectors.  For example, {\tt dotProduct([1,2,3], [4,5,6])} should return {\tt 32} (since that is {\tt 1*4 + 2*5 + 3*6}).  You may assume that the two argument lists have the same length.

\bigskip

\begin{verbatim}
def dotProduct(vec1, vec2):
    # your code goes here
\end{verbatim}

\cprotect
\IfSolnElse{
\begin{verbatim}
def dotProduct(vec1, vec2):
    if vec1 == []:
        return 0
    else:
        return vec1[0] * vec2[0] + dotProduct(vec1[1:], vec2[1:])
\end{verbatim}
}{}

\end{enumerate}
\end{document}