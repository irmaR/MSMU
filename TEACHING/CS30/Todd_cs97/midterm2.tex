\documentclass[11pt]{article}

\usepackage{ifthen}
\usepackage{cprotect}

\newboolean{Solutions}
\setboolean{Solutions}{true}
\newcommand{\IfSolnElse}[2]{\ifthenelse{\boolean{Solutions}}{#1}{#2}}

\topmargin = 0in
\oddsidemargin = 0in
\evensidemargin = \oddsidemargin
\textwidth = 6.5in
\textheight = 9in

\title{Midterm Exam \#2}

\author{CS97: Principles and Practices of Computing}

\date{Wednesday, November 15, 2017}

\begin{document}  
\maketitle
\thispagestyle{empty}
\pagestyle{empty}

  \begin{flushright}
\fbox{  
\begin{tabular}{ll}
  1. \hspace{.5in} & 2. \hspace{.5in} \\\\\\
  3.& 4. \\\\\\
\end{tabular}
}
\end{flushright}

\vspace{.5in}

Name:$\underline{\mbox{\hspace{4in}}}$

\vspace{.5in}

ID:$\underline{\mbox{\hspace{4.2in}}}$

\vspace{0.5in}

Rules of the game:

\begin{itemize}
  \item {\bf Write your name and ID number above.}
\item The exam is closed-book and closed-notes.

\item Please write your answers
directly on the exam.  Do not turn in anything else.

% \item Obey our usual OCaml style rules.

% \item Except where explicitly disallowed, you can write any number of
 % helper functions that you need.

% \item If you have any questions, please ask.  
  
\item The exam ends promptly at 3:50pm.
  
\item Read questions carefully.  Understand a question before you
  start writing. %   {\em Note:  Some multiple-choice
    % questions ask for a single answer, while others ask for all
    % appropriate answers.} 

\item Relax!
\end{itemize}

\begin{enumerate}

\pagebreak

\item (2 points each)
The {\em Fibonacci sequence} of nonnegative numbers is defined as follows.  The $0$th Fibonacci number is $0$, the $1$st Fibonacci number is $1$, and the $n$th Fibonacci number (where $n > 1$) is the sum of the previous two Fibonacci numbers.

Here is a Python function that returns the $n$th number in the Fibonacci sequence:

\begin{verbatim}
def fib(n):
    if n == 0:
        return 0
    elif n == 1:
        return 1
    else:
        return fib(n-1) + fib(n-2)
\end{verbatim}

\bigskip

\begin{enumerate}
\item What is the result of {\tt fib(4)}?

\IfSolnElse{3}{\vspace{1.5in}}

\item What are the two argument integers to the first addition operation (i.e., execution of the {\tt +} operator) that happens during the execution of {\tt fib(4)}?  Provide the integers in order from left to right.

\IfSolnElse{1 and 0}{\vspace{1.5in}}

\item How many times is {\tt fib(1)} executed during the execution of {\tt fib(4)}?

\IfSolnElse{3}{\vspace{1.5in}}

\end{enumerate}

\pagebreak

\item (5 points) Implement a function {\tt makeList} that takes a function {\tt f} and a nonnegative integer {\tt n} and returns the list {\tt [f(0),$\ldots$,f(n-1)]}.  For example, {\tt makeList(lambda x: 2*x, 5)} 
returns {\tt [0, 2, 4, 6, 8]} and {\tt makeList(lambda x: 2*x, 0)} returns {\tt []}.

\cprotect
\IfSolnElse{
\begin{verbatim}
def makeList(f, n):
    if n == 0:
        return []
    else:
        return makeList(f, n-1) + [f(n-1)]
\end{verbatim}
}{}

\pagebreak

\item (2 points each)  {\bf Circle all answers that are true for each question.}  MAP AND FILTER SHOULD HAVE LIST(...) AROUND THEM!

\begin{enumerate}

\item Which functions 
{\em always} return a list of length less than that of the argument list?
\begin{enumerate}
\item {\tt map}
\item {\tt filter}
\item {\tt reduce}
\item none of the above
\end{enumerate}
\IfSolnElse{iv}{\vspace{.2in}}

\item Which functions {\em always} return a list of the same type of elements as the argument list (e.g., given an integer list, the function will always return an integer list)?  CONFUSING BECAUSE WHAT IS THE TYPE OF THE EMPTY LIST?
\begin{enumerate}
\item {\tt map}
\item {\tt filter}
\item {\tt reduce}
\item none of the above
\end{enumerate}
\IfSolnElse{ii}{\vspace{.2in}}

\item Which functions {\em always} return a list?
\begin{enumerate}
\item {\tt map}
\item {\tt filter}
\item {\tt reduce}
\item none of the above
\end{enumerate}
\IfSolnElse{i and ii}{\vspace{.2in}}

\item Which functions {\em never} return a list?
\begin{enumerate}
\item {\tt map}
\item {\tt filter}
\item {\tt reduce}
\item none of the above
\end{enumerate}
\IfSolnElse{iv}{\vspace{.2in}}

\end{enumerate}

\pagebreak

\item (5 points) Implement a function {\tt swapPairs} that takes a list of pairs and returns an identical list but with the elements of each pair swapped.  For example,
\begin{center}
{\tt swapPairs([[1, "one"], [2, "two"], [3, "three"]])} 
\end{center}
returns 
{\tt [["one", 1], ["two", 2], ["three", 3]]}.

{\bf You may not use recursion for this problem.  Instead, make appropriate use of {\tt map}, {\tt filter}, and/or {\tt reduce}.}

\cprotect
\IfSolnElse{
\begin{verbatim}
def swapPairs(alist):
    return list(map(lambda p: [p[1], p[0]], alist))
\end{verbatim}
}{}

\end{enumerate}
\end{document}